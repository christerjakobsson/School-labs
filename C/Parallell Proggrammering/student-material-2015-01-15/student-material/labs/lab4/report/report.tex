\documentclass[a4paper,12pt]{article}

\usepackage[utf8]{inputenc}
\usepackage[T1]{fontenc}
\usepackage{amsmath,amssymb,amsfonts}
\usepackage{graphicx}
\usepackage[margin=20mm]{geometry}
\usepackage{color}

\newcommand{\todo}[1]{{\color{blue} TODO[#1]}}

\begin{document}

\title{5DV152/VT15: Lab 4}
\author{\todo{Your Name} (\todo{Your Personnumber})}
\date{\todo{Submission Date}}
\maketitle


\section{Experiment design}
\label{sec:experiment-design}

Describe the details of the experiment design, including
\begin{itemize}
\item The size of the image.
\item The pixel size.
\item The center point in the complex plane.
\item The maximum number of iterations.
\item The set of thread counts used.
\item The metric used to quantify the load imbalance.
\end{itemize}

\section{Results}
\label{sec:results}

Summarize the results, including
\begin{itemize}
\item One or two plots showing the mean execution times for both the static and dynamic scheduling cases as a function of the number of threads.
\item One or two plots showing the mean speed-up for both the static and dynamic scheduling cases as a function of the number of threads including error bars.
\item A plot showing the mean load imbalance of the static scheduling case as a function of the number of threads including error bars.
\end{itemize}

\section{Conclusion}
\label{sec:conclusion}

Discuss the results, including
\begin{itemize}
\item A discussion of the scalability of the static and dynamic scheduling cases.
\item A quantitative and qualitative comparison of the scalability of the static and dynamic scheduling cases.
\item Reflections on the behavior of the load imbalance.
\end{itemize}
  
\end{document}
