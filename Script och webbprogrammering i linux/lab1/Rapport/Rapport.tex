
\documentclass[11pt, titlepage, oneside, a4paper]{article}
\usepackage[T1]{fontenc}
\usepackage[utf8]{inputenc}
\usepackage[english]{babel}
\usepackage{amssymb, graphicx, fancyhdr}
\usepackage{listings}
\lstset{breaklines=true} 
\lstset{numbers=left, numberstyle=\scriptsize\ttfamily, numbersep=10pt, captionpos=b} 
\lstset{basicstyle=\small\ttfamily}
\newcommand{\inlineCode}{\lstinline[basicstyle=\normalsize\ttfamily]}
\addtolength{\textheight}{20mm}
\addtolength{\voffset}{-5mm}
\renewcommand{\sectionmark}[1]{\markleft{#1}}

% \Section ger mindre spillutrymme, använd dem om du vill
\newcommand{\Section}[1]{\section{#1}\vspace{-8pt}}
\newcommand{\Subsection}[1]{\vspace{-4pt}\subsection{#1}\vspace{-8pt}}
\newcommand{\Subsubsection}[1]{\vspace{-4pt}\subsubsection{#1}\vspace{-8pt}}
	
% appendices, \appitem och \appsubitem är för bilagor
\newcounter{appendixpage}

\newenvironment{appendices}{
	\setcounter{appendixpage}{\arabic{page}}
	\stepcounter{appendixpage}
}

\newcommand{\appitem}[2]{
	\stepcounter{section}
	\addtocontents{toc}{\protect\contentsline{section}{\numberline{\Alph{section}}#1}{\arabic{appendixpage}}}
	\addtocounter{appendixpage}{#2}
}

\newcommand{\appsubitem}[2]{
	\stepcounter{subsection}
	\addtocontents{toc}{\protect\contentsline{subsection}{\numberline{\Alph{section}.\arabic{subsection}}#1}{\arabic{appendixpage}}}
	\addtocounter{appendixpage}{#2}
}

% Ändra de rader som behöver ändras
\def\inst{Tillämpad fysik och elektronik}
\def\typeofdoc{Laborationsrapport}
\def\course{Script och webbprogrammering 7,5 hp}
\def\pretitle{Laboration 1}
\def\title{Scriptprogrammering}
\def\name{Christer Jakobsson}
\def\username{dv12cjn}
\def\email{\username{}@cs.umu.se}
\def\graders{Ola Ågren, Kalle Prorok}



% Här brjar själva dokumentet
\begin{document}

	% Skapar framsidan (om den inte duger: gör helt enkelt en egen)
	\begin{titlepage}
		\thispagestyle{empty}
		\begin{large}
			\begin{tabular}{@{}p{\textwidth}@{}}
				\textbf{UMEÅ UNIVERSITET \hfill \today} \\
				\textbf{Institutionen för \inst} \\
				\textbf{\typeofdoc} \\
			\end{tabular}
		\end{large}
		\vspace{10mm}
		\begin{center}
			\LARGE{\pretitle} \\
			\huge{\textbf{\course}}\\
			\vspace{10mm}
			\LARGE{\title} \\
			\vspace{15mm}
			\begin{large}
				\begin{tabular}{ll}
					\textbf{Namn} & \name \\
					\textbf{E-mail} & \texttt{\email} \\
				\end{tabular}
			\end{large}
			\vfill
			\large{\textbf{Handledare}}\\
			\mbox{\large{\graders}}
		\end{center}
	\end{titlepage}


	% Fixar sidfot
	\lfoot{\footnotesize{\name, \email}}
	\rfoot{\footnotesize{\today}}
	\lhead{\sc\footnotesize\title}
	\rhead{\nouppercase{\sc\footnotesize\leftmark}}
	\pagestyle{fancy}
	\renewcommand{\headrulewidth}{0.2pt}
	\renewcommand{\footrulewidth}{0.2pt}

	
	
	\pagenumbering{arabic}
	\tableofcontents
	\newpage
	% I Sverige har vi normalt inget indrag vid nytt stycke
	\setlength{\parindent}{0pt}
	% men däremot lite mellanrum
	\setlength{\parskip}{10pt}

	\Section{Del 1: Analys av befintligt script}
	Denna fil generar en sträng som är av längd 8 och har slumpade tecken ifrån hela alfabetet, stora som små bokstäver.
	Scriptet initierar en variabel (\emph{MATRIX}) som innehåller alla tecken i alfabetet i stora och små bokstäver.
	och loopar 8 gånger och plockar en random bokstav ifrån \emph{MATRIX} och lägger till den i slutet på en variabel som heter \emph{PASS}.
	Därefter så skrivs variabeln \emph{PASS}:s data ut.

	I och med att variabeln som skrivs ut på skärmen som resultat heter \emph{PASS} så gissar jag att detta program är tänkt att skapa ett slumpgenererat lösenord
	och namnet torde stå för \textbf{new password} (np).
	
	\Section{Del 2: Enklare uppgift}
	\Subsubsection{Users}
	För att ta reda på vilka användare som finns på datorn så använder jag mig av filen passwd som innehåller information om varje användare, filen ligger i 
	\emph{/etc/passwd}. Så genom att köra \emph{cat /etc/passwd} och sedan pipa vidare det till \emph{cut -d -f1} som tar bort all annan data i filen förutom användarnamnen,
	så ger detta utdata med användarnamn i rader. Därefter så pipas detta vidare till \emph{pr -s' ' -5 -l1 -t} som gör så att, ifrån det varit ett användarnamn per rad till att det
	är 5 användarnamn per rad. Därefter pipas detta vidare till \emph{awk '{printf "\%10s  \%10s  \%10s  \%10s  \%10s\char`\\n", \$1, \$2, \$3, \$4, \$5}'} som skriver till standard out
	med 10 som mellanrum emellan varje användarnamn, detta skrivs sedan ut på standard out.
	\newline
	Kommandot i sin helhet: \newline\textbf{cat /etc/passwd | cut -d: -f1 | pr -s' ' -5 -l1 -t | awk '{printf "\%10s  \%10s  \%10s  \%10s  \%10s\char`\\n", \$1, \$2, \$3, \$4, \$5}'}
	
	Ex:
	\begin{verbatim}
	       root      daemon         bin         sys        sync
     games         man          lp        mail        news
      uucp       proxy    www-data      backup        list
       irc       gnats      nobody     libuuid      syslog
messagebus      usbmux     dnsmasq  avahi-autoipd    kernoops
     rtkit       saned    whoopsie  speech-dispatcher       avahi
   lightdm      colord       hplip       pulse     shinowa

	\end{verbatim}

	\Subsubsection{Groups}
	Genom att använda kommandot \emph{groups} så skrivs grupperna som finns på datorn ut på standard out. 
	
	Ex:
	\begin{verbatim}
	 shinowa adm cdrom sudo dip plugdev lpadmin sambashare
	\end{verbatim}

	\Subsubsection{Användare inloggade just nu}
	Genom kommandot \emph{users} så visas de användare som är inloggade för tillfället ut på standard out.
	
	Ex:
	\begin{verbatim}
	 shinowa shinowa
	\end{verbatim}

	
	\Section{Del 3: Något mer tillkrånglad uppgift}	
	För att visa biblioteksstrukturen från en viss startaddress har jag gjort att användaren får skriva startaddressen som programmets första arguemnt. Sedan så använder jag kommandot \emph{find \$1} där \emph{\$1} är scriptets första argument för att skriva ut hela strukturen.
	Ex:
	\begin{verbatim}
.
./np
./usergroup
./libscript
./Rapport
./Rapport/Rapport.tex.backup
./Rapport/Rapport.aux
./Rapport/Rapport.pdf
./Rapport/Rapport.tex
./Rapport/.Rapport.tex.kate-swp
./Rapport/Rapport.log

	\end{verbatim}

	Sedan för att formatera utdatan så den liknar 
	datan ifrån \emph{tree} så pipar jag vidare till \begin{verbatim}
\emph{sed -e "s;\$pwd;\char`\\.;g;s;[^/]*\/;|__;g;s;__|; |;g"}
	                                                 \end{verbatim}
 som kommer att strukturera varje rad så att man kan 
	se hur djupt ner filen eller mappen är. För varje / så kommer namnet på filen att ha ett |\_\_ extra framför sig och om filen är en mapp så kommer det att vara ett | under namnet som symboliserar att
	filerna under ligger i den mappen.
	
	Ex:
	\begin{verbatim}
.
|__np
|__usergroup
|__libscript
|__Rapport
| |__Rapport.tex.backup
| |__Rapport.aux
| |__Rapport.pdf
| |__Rapport.tex
| |__Rapport.log
	\end{verbatim}

	Detta liknar tree till utseendet men använder sig bara av standardprogrammmen \emph{find} och \emph{sed}.
	
	\Subsubsection{Användarhandledning}
	Programmet heter \emph{libscript} och för att köra programmet så öppnar man en terminal i den katalog som scriptet ligger i och skriver \emph{./libscript pathToList}.
	\emph{libscript} tar ett argument som är addressen för den katalog som det ska visa strukturen för.
	
	Ex: \begin{verbatim}
./libscript /home/shinowa/Dropbox/scriptprog/lab1
.
|__usergroup
|__libscript
|__Rapport
| |__Rapport.tex.backup
| |__Rapport.aux
| |__Rapport.toc
| |__Rapport.pdf
| |__Rapport.tex
| |__.Rapport.tex.kate-swp
| |__Rapport.log

	    \end{verbatim}

	
	\Section{Källkod}
	\emph{här följer källkoden för de två uppgifterna där script skulle skapas.}
	\Subsubsection{Del 2: Enklare uppgift}

	\begin{lstlisting}
#!/bin/bash

printf "\n----------------Users-----------------------\n"
cat /etc/passwd | cut -d: -f1 | pr -s' ' -5 -l1 -t | awk '{printf "%10s  %10s  %10s  %10s  %10s\n", $1, $2, $3, $4, $5}' 
printf "\n----------------End------------------------\n"

printf "\n----------------Groups-----------------------\n"
groups

printf "\n----------------Users logged in on computer-----------------------\n"
users
\end{lstlisting}
	
	\Subsubsection{Del 3: Något mer tillkrånglad uppgift}
	\begin{lstlisting}
#!/bin/bash

if [ "$#" -ne 1 ];
then
    printf "$0 takes a directory to list as argument, exiting...\n"
    exit 1
fi

find $1 | sed -e "s;$1;\.;g;s;[^/]*\/;|__;g;s;__|; |;g"
	\end{lstlisting}

	
	% Lägger in rubrik (finns \section, men då får man mycket spillutrymme)
	\end{document}