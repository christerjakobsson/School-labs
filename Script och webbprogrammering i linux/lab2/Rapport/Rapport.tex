
\documentclass[11pt, titlepage, oneside, a4paper]{article}
\usepackage[T1]{fontenc}
\usepackage[utf8]{inputenc}
\usepackage[english]{babel}
\usepackage{amssymb, graphicx, fancyhdr}
\usepackage{listings}
\lstset{breaklines=true} 
\lstset{numbers=left, numberstyle=\scriptsize\ttfamily, numbersep=10pt, captionpos=b} 
\lstset{basicstyle=\small\ttfamily}
\newcommand{\inlineCode}{\lstinline[basicstyle=\normalsize\ttfamily]}
\addtolength{\textheight}{20mm}
\addtolength{\voffset}{-5mm}
\renewcommand{\sectionmark}[1]{\markleft{#1}}

% \Section ger mindre spillutrymme, använd dem om du vill
\newcommand{\Section}[1]{\section{#1}\vspace{-8pt}}
\newcommand{\Subsection}[1]{\vspace{-4pt}\subsection{#1}\vspace{-8pt}}
\newcommand{\Subsubsection}[1]{\vspace{-4pt}\subsubsection{#1}\vspace{-8pt}}
	
% appendices, \appitem och \appsubitem är för bilagor
\newcounter{appendixpage}

\newenvironment{appendices}{
	\setcounter{appendixpage}{\arabic{page}}
	\stepcounter{appendixpage}
}

\newcommand{\appitem}[2]{
	\stepcounter{section}
	\addtocontents{toc}{\protect\contentsline{section}{\numberline{\Alph{section}}#1}{\arabic{appendixpage}}}
	\addtocounter{appendixpage}{#2}
}

\newcommand{\appsubitem}[2]{
	\stepcounter{subsection}
	\addtocontents{toc}{\protect\contentsline{subsection}{\numberline{\Alph{section}.\arabic{subsection}}#1}{\arabic{appendixpage}}}
	\addtocounter{appendixpage}{#2}
}

% Ändra de rader som behöver ändras
\def\inst{Tillämpad fysik och elektronik}
\def\typeofdoc{Laborationsrapport}
\def\course{Script och webbprogrammering 7,5 hp}
\def\pretitle{Laboration 2}
\def\title{Scriptprogrammering}
\def\name{Christer Jakobsson}
\def\username{dv12cjn}
\def\email{\username{}@cs.umu.se}
\def\graders{Ola Ågren, Kalle Prorok}



% Här brjar själva dokumentet
\begin{document}

	% Skapar framsidan (om den inte duger: gör helt enkelt en egen)
	\begin{titlepage}
		\thispagestyle{empty}
		\begin{large}
			\begin{tabular}{@{}p{\textwidth}@{}}
				\textbf{UMEÅ UNIVERSITET \hfill \today} \\
				\textbf{Institutionen för \inst} \\
				\textbf{\typeofdoc} \\
			\end{tabular}
		\end{large}
		\vspace{10mm}
		\begin{center}
			\LARGE{\pretitle} \\
			\huge{\textbf{\course}}\\
			\vspace{10mm}
			\LARGE{\title} \\
			\vspace{15mm}
			\begin{large}
				\begin{tabular}{ll}
					\textbf{Namn} & \name \\
					\textbf{E-mail} & \texttt{\email} \\
				\end{tabular}
			\end{large}
			\vfill
			\large{\textbf{Handledare}}\\
			\mbox{\large{\graders}}
		\end{center}
	\end{titlepage}


	% Fixar sidfot
	\lfoot{\footnotesize{\name, \email}}
	\rfoot{\footnotesize{\today}}
	\lhead{\sc\footnotesize\title}
	\rhead{\nouppercase{\sc\footnotesize\leftmark}}
	\pagestyle{fancy}
	\renewcommand{\headrulewidth}{0.2pt}
	\renewcommand{\footrulewidth}{0.2pt}

	
	
	\pagenumbering{arabic}
	\tableofcontents
	\newpage
	% I Sverige har vi normalt inget indrag vid nytt stycke
	\setlength{\parindent}{0pt}
	% men däremot lite mellanrum
	\setlength{\parskip}{10pt}

	\Section{Del 1: Analys av befintligt script}
	
	Detta script gör om utdatan till att beroende på hur djupt ner i katalogstrukturen som något ligger så får dess utskrift ifrån ls ett mellanslag framför sig.
	Om ls listar katalogerna i \emph{/home/användare} så kommer sed att göra så all utskrift ifrån ls rörande denna katalog kommer ha tre mellanslag framför sig.
	
	När jag kör ls -lR och pipar detta till scriptet så kommer ls att rekursivt lista alla undermappar, och scriptet kommer att lägga till mellanslag så man enklare kan se katalogstrukturen.
	
	Eftersom att programmet ändrar utskriften på katalogstruckturen så att man kan see hur trädstrukturen  ser ut så skulle scriptet kunna heta \emph{tabulateLsOutput}.
	
	
	\Section{Del 2: awk-uppgift.}
	
	Sed scriptet \emph{sed -e '/:\$/d' /etc/group | cut -d: -f1} skriver ut de rader där det finns ett värde efter sista avgränsaren. Formatet på filen \emph{/etc/group} är: \texttt{Group name:Password:Group ID:Group List}.
	Så de rader där \emph{Group List} innehåller ett värde är de som scriptet pipar vidare till \emph{cut -d: -f1} som tar bort alla värden i raden och lämnar kvar \emph{Group name} som skrivs ut.
	
	Mitt awk program fungerar så att jag säger att avgränsaren ska vara \texttt{:} så awk kommer att dela upp raden i fyra stycken delar, och om del fyra inte är tom så ska raden skrivas ut.
	
	\Subsection{Källkod}
	\begin{lstlisting}
awk -F ':' '{if ($4 != "")  print $1}' /etc/group
	\end{lstlisting}

	

	
	
	\Section{Del 3: perl-uppgift}	
	Detta perl script ska visa biblioteksstrukturen för en viss startaddress. Det får inte använda sig av några standardprogram.
	\Subsection{Användarhandledning}
	För att köra scriptet så öppnar man en terminal i den katalog som man har scriptet och skriver \texttt{./[scriptnamn].pl [\emph{pathToSearch}]}. där \emph{scriptnamn} är det namn man har gett scriptet
, och pathToSearch är den sökväg man vill visa biblioteksstrukturen på. Programmet kan behövas göra exekverbart och detta görs genom att skriva texttt{chmod 777 [scriptnamn].pl}

Exempelkörning:
\begin{verbatim}
 /home/shinowa/Dropbox/scriptprog//lab3
/home/shinowa/Dropbox/scriptprog//lab3/awkuppgift~
/home/shinowa/Dropbox/scriptprog//lab2
/home/shinowa/Dropbox/scriptprog//lab2/libscript.pl~
/home/shinowa/Dropbox/scriptprog//lab2/libscript~
/home/shinowa/Dropbox/scriptprog//lab2/awkuppgift
/home/shinowa/Dropbox/scriptprog//lab2/namn~
/home/shinowa/Dropbox/scriptprog//lab2/Rapport
/home/shinowa/Dropbox/scriptprog//lab2/Rapport/Rapport.tex.backup
/home/shinowa/Dropbox/scriptprog//lab2/Rapport/Rapport.aux
/home/shinowa/Dropbox/scriptprog//lab2/Rapport/Rapport.toc
/home/shinowa/Dropbox/scriptprog//lab2/Rapport/Rapport.pdf
/home/shinowa/Dropbox/scriptprog//lab2/Rapport/Rapport.tex
/home/shinowa/Dropbox/scriptprog//lab2/Rapport/Rapport.log
/home/shinowa/Dropbox/scriptprog//lab2/libscript.pl
/home/shinowa/Dropbox/scriptprog//lab1
/home/shinowa/Dropbox/scriptprog//lab1/Rapport.pdf
/home/shinowa/Dropbox/scriptprog//lab1/usergroup
/home/shinowa/Dropbox/scriptprog//lab1/libscript
/home/shinowa/Dropbox/scriptprog//lab1/Rapport
/home/shinowa/Dropbox/scriptprog//lab1/Rapport/Rapport.tex.backup
/home/shinowa/Dropbox/scriptprog//lab1/Rapport/Rapport.aux
/home/shinowa/Dropbox/scriptprog//lab1/Rapport/Rapport.toc
/home/shinowa/Dropbox/scriptprog//lab1/Rapport/Rapport.pdf
/home/shinowa/Dropbox/scriptprog//lab1/Rapport/Rapport.tex
/home/shinowa/Dropbox/scriptprog//lab1/Rapport/Rapport.log
\end{verbatim}

	
	
	
	\Subsection{Källkod}
	\begin{lstlisting}
#!/usr/bin/perl

use strict;
use warnings;

my $dir = $ARGV[0];

traverse($dir);    
    
    sub traverse {
	my ($thing) = @_;

	return if not -d $thing;
	
	opendir(my $DIR, $thing) or die $!;    
	while (my $file = readdir($DIR)) {
	
	    # A file test to check that it is a directory
	    # Use -f to test for a file
	    
	    next if $file eq '.' or $file eq '..';
	    print "$thing/$file\n";
	    traverse("$thing/$file");
	}

	closedir($DIR);
    }
exit 0;
	\end{lstlisting}

	
		\end{document}