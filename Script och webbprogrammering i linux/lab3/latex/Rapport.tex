
\documentclass[11pt, titlepage, oneside, a4paper]{article}
\usepackage[T1]{fontenc}
\usepackage[utf8]{inputenc}
\usepackage[english]{babel}
\usepackage{amssymb, graphicx, fancyhdr}
\usepackage{listings}
\lstset{breaklines=true} 
\lstset{numbers=left, numberstyle=\scriptsize\ttfamily, numbersep=10pt, captionpos=b} 
\lstset{basicstyle=\small\ttfamily}
\newcommand{\inlineCode}{\lstinline[basicstyle=\normalsize\ttfamily]}
\addtolength{\textheight}{20mm}
\addtolength{\voffset}{-5mm}
\renewcommand{\sectionmark}[1]{\markleft{#1}}

% \Section ger mindre spillutrymme, använd dem om du vill
\newcommand{\Section}[1]{\section{#1}\vspace{-8pt}}
\newcommand{\Subsection}[1]{\vspace{-4pt}\subsection{#1}\vspace{-8pt}}
\newcommand{\Subsubsection}[1]{\vspace{-4pt}\subsubsection{#1}\vspace{-8pt}}
	
% appendices, \appitem och \appsubitem är för bilagor
\newcounter{appendixpage}

\newenvironment{appendices}{
	\setcounter{appendixpage}{\arabic{page}}
	\stepcounter{appendixpage}
}

\newcommand{\appitem}[2]{
	\stepcounter{section}
	\addtocontents{toc}{\protect\contentsline{section}{\numberline{\Alph{section}}#1}{\arabic{appendixpage}}}
	\addtocounter{appendixpage}{#2}
}

\newcommand{\appsubitem}[2]{
	\stepcounter{subsection}
	\addtocontents{toc}{\protect\contentsline{subsection}{\numberline{\Alph{section}.\arabic{subsection}}#1}{\arabic{appendixpage}}}
	\addtocounter{appendixpage}{#2}
}

% Ändra de rader som behöver ändras
\def\inst{Tillämpad fysik och elektronik}
\def\typeofdoc{Laborationsrapport}
\def\course{Script och webbprogrammering 7,5 hp}
\def\pretitle{Laboration 3}
\def\title{Python}
\def\name{Christer Jakobsson}
\def\username{dv12cjn}
\def\email{\username{}@cs.umu.se}
\def\graders{Ola Ågren, Kalle Prorok}



% Här brjar själva dokumentet
\begin{document}

	% Skapar framsidan (om den inte duger: gör helt enkelt en egen)
	\begin{titlepage}
		\thispagestyle{empty}
		\begin{large}
			\begin{tabular}{@{}p{\textwidth}@{}}
				\textbf{UMEÅ UNIVERSITET \hfill \today} \\
				\textbf{Institutionen för \inst} \\
				\textbf{\typeofdoc} \\
			\end{tabular}
		\end{large}
		\vspace{10mm}
		\begin{center}
			\LARGE{\pretitle} \\
			\huge{\textbf{\course}}\\
			\vspace{10mm}
			\LARGE{\title} \\
			\vspace{15mm}
			\begin{large}
				\begin{tabular}{ll}
					\textbf{Namn} & \name \\
					\textbf{E-mail} & \texttt{\email} \\
				\end{tabular}
			\end{large}
			\vfill
			\large{\textbf{Handledare}}\\
			\mbox{\large{\graders}}
		\end{center}
	\end{titlepage}


	% Fixar sidfot
	\lfoot{\footnotesize{\name, \email}}
	\rfoot{\footnotesize{\today}}
	\lhead{\sc\footnotesize\title}
	\rhead{\nouppercase{\sc\footnotesize\leftmark}}
	\pagestyle{fancy}
	\renewcommand{\headrulewidth}{0.2pt}
	\renewcommand{\footrulewidth}{0.2pt}

	
	
	\pagenumbering{arabic}
	\tableofcontents
	\newpage
	% I Sverige har vi normalt inget indrag vid nytt stycke
	\setlength{\parindent}{0pt}
	% men däremot lite mellanrum
	\setlength{\parskip}{10pt}
	
	\Section{Problemspecifikation}
	Denna uppgift har gått ut på att skapa ett script i programspråket Python som tar ett godtyckligt antal argument, där argumenten ska vara sökvägar till mappar. 
	I dessa sökvägar så ska alla \emph{.txt} filer i mappens två första rader skrivas ut på standard out.
	
	Exempel:
	\begin{verbatim}
	 output.txt:
 Hello world this is on the same line
 This is some text
python.txt:
 (:title A quick introduction to Python:)
python2.txt:
 After having [[Python.Strings|learned about strings]]
 it's pretty easy to understand how we can remove the newline from the
t1.txt:
 Lorem ipsum dolor sit amet, consectetuer adipiscing elit. 
 Integer mi dui, varius non, fermentum eget, adipiscing in,
t2.txt:
 Lorem ipsum dolor sit amet, consectetuer adipiscing elit.
 Integer mi dui, varius non, fermentum eget, adipiscing in, 
	\end{verbatim}

	\Section{Användarhandledning}
	Scriptet heter \emph{uppgift.py} och för att köra det så öppnar man en terminal i den mapp som scriptet ligger och skriver \newline
	\texttt{python uppgift.py \emph{sökväg1 sökväg2 ...}}
	
	Programmet kommer då att skriva ut de två första raderna i varje \emph{.txt} i mappen, förutsatt att användaren har rättigheter att öppna filen.
	\Section{Källkod}
	\begin{lstlisting}
#!/usr/bin/python

import sys
import glob
import os

# Function that opens all the .txt files in the directories given as arguments
# and prints the two first lines in them to standard out.
def openAndReadTxtFiles():
	lista = []
		
	for i in range(1, len(sys.argv)):
	   	lista.append(sys.argv[i])
	
	for path in lista:
      		for filename in glob.glob(os.path.join(path, '*.txt')):
      			openFile = open(filename, 'r')
			print os.path.basename(filename)
			print "\t"+openFile.readline()[:-2]
			print "\t"+openFile.readline()[:-2]
	return

# If there is more then one argument to the program it run openAndReadTxtFiles.
if (len(sys.argv) <= 1):
    sys.stderr.write('Not enough arguments, exiting...\n')
    raise SystemExit(1)
else:    
	openAndReadTxtFiles()

	\end{lstlisting}

  
		\end{document}